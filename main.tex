\documentclass[11pt]{article}
\usepackage{graphicx} % Required for inserting images
\usepackage[utf8]{inputenc}
\usepackage{amsmath, amssymb, mathtools}
\usepackage{physics}
\usepackage{braket}
\usepackage{geometry}
\usepackage{setspace}
\usepackage{sectsty}
\usepackage{titling}
\usepackage{tocloft}
\usepackage{hyperref}
\geometry{a4paper,  margin=1in}
\setstretch{1.5}
\numberwithin{equation}{section}
\AtBeginDocument{\Large \selectfont}
\title{Canonical Quantization Notes}
\author{YILUN LIN}
\date{October 2025}

\begin{document}

\maketitle
\Large\textbf{Principle}
\newline
\Large{The central idea of canonical quantization is to construct a noncommutative operator algebra that deforms the classical algebra of observables.}
\section{Procedure of Canonical Quantization}
\begin{enumerate}
    \item Specify the dynamics of the classical system.
    \begin{enumerate}
        \item Construct the Lagrangian (density) $\mathcal{L}\left[\phi_{\nu}\left(\mathbf{x}\right),\,\partial_{\mu}\phi_{\nu}\left(\mathbf{x}\right)\right]$ of the classical field.
        \item The Lagrangian must include a kinetic term and a potential term. 
        \item The kinetic term describes the motion and propagation of the field. It must explicitly depend on the field's derivatives, ensuring that the field has proper propagating degrees of freedom. Crucially, the kinetic term must be a Lorentz scalar, which means that the structure must stay the same under Lorentz transformation. Actually, every term in the Lagrangian must be a Lorentz scalar. 
        \item The potential term is composed of a mass term and a self-interaction term. The potential term must not include derivatives. The quadratic term in the fields is the mass term, and the terms that are cubic or higher order in the fields describe self-interaction. Also, each term should be a Lorentz scalar. 
        \item The global symmetry of the theory should be considered while constructing the Lagrangean. For example, because quantum electrodynamics possesses $U\left(1\right)$ invariance, the Lagrangean of it must be invariant under $U\left(1\right)$ global transformation, or even introduce the corresponding gauge field $A_{\mu}$ to achieve $U\left(1\right)$ gauge symmetry if needed. This is because that the gauge symmetry is optional in classical field theory, but it is required in quantum field theory. 
        \item For example, the Lagrangean of the simplest field \--\-- Klein-Gorden Field ($spin=0$ particle field) \--\-- is $\mathcal{L}\,=\,\frac{1}{2}\partial_{\mu}\phi\,\partial^{\mu}\phi-\frac{1}{2}m^{2}\phi^{2}$.
        \newline Apply the Euler-Lagrangean equation:
        \begin{equation}
            \begin{split}
                &\frac{\partial\mathcal{L}}{\partial\phi}\,-\,\partial_{\mu}\left[\frac{\partial\mathcal{L}}{\partial\left(\partial_{\mu}\phi\right)}\right]\,=\,0 \\
                \longrightarrow&\left(\Box+m^{2}\right)\phi\,=\,0.
            \end{split}
        \end{equation}
    \end{enumerate}
    \item Determine the conjugate momenta and calculate the Hamiltonian.
    \begin{enumerate}
        \item The canonical momentum is defined as 
        \begin{equation}\label{eq: definition of canonical momentum}
            \begin{split}
               \mathcal{\pi}_{\mu}\left[\phi_{\nu}\left(\mathbf{x}\right),\,\partial_{\alpha}\phi_{\nu}\left(\mathbf{x}\right)\right]\,&=\,\frac{\partial\mathcal{L}}{\partial\left[\partial_{0}\phi_{\mu}\left(\mathbf{x}\right)\right]}.
            \end{split}
        \end{equation}
        \item Derive the Hamiltonian of the classical field through Legendre transformation, that is,
        \begin{equation}
            \begin{split}
                \mathcal{H}\left[\phi_{\nu}\left(\mathbf{x}\right),\,\partial_{\alpha}\phi_{\nu}\left(\mathbf{x}\right)\right]\,&=\,\mathcal{\pi}_{\mu}\left[\phi_{\nu}\left(\mathbf{x}\right),\,\partial_{\alpha}\phi_{\nu}\left(\mathbf{x}\right)\right]\partial_{0}\phi^{\mu}\left(\mathbf{x}\right)\,-\,\mathcal{L}\left[\phi_{\nu}\left(\mathbf{x}\right),\,\partial_{\alpha}\phi_{\nu}\left(\mathbf{x}\right)\right].
            \end{split}
        \end{equation}
        \item But now the explicit parameters of $\mathcal{H}$ are fields and their (both time and spatial) derivatives. In order to undergo canonical quantization, the explicit parameters of $\mathcal{H}$ must be $\left[\phi_{\nu},\,\pi_{\nu},\,\nabla\phi_{\nu}\right]$. That is, time derivatives of the field should be replaced with corresponding canonical momenta. From \eqref{eq: definition of canonical momentum}, we can obtain the relations $\partial_{0}\phi_{\nu}=\dot{\phi}_{\nu}\left[\phi_{\mu},\,\mathcal{\pi}_{\mu},\,\nabla\phi_{\mu}\right]$, with the time derivatives $\partial_{0}\phi_{\mu}$ replaced by the same amount of independent canonical momenta $\mathcal{\pi}_{\mu}$. 
    \end{enumerate}
    \item Analyze constraints in the dynamics of the classical system.
    \begin{enumerate}
        \item In order to determine whether there are constraints in the system, we need to observe the Hessian matrix:
        \begin{equation}
            \begin{split}
                H_{\mu\nu}\,:&=\,\frac{\partial^{2}\mathcal{L}}{\partial\left(\partial_{0}\phi^{\mu}\right)\,\partial\left(\partial_{0}\phi^{\nu}\right)}.
            \end{split}
        \end{equation}
        If the Hessian matrix has full rank, then there is no constraint in the system. Otherwise, some constraints may exist. 
        \item Primary constraints: If the Hessian matrix is not invertible, some time derivatives cannot be solved, using relations \eqref{eq: definition of canonical momentum}, as a unique result in terms of canonical momenta. Those constraints fall into two types:
        \begin{equation}\label{eq: constraints on momenta}
            \begin{split}
                F\left(\phi^{\mu},\,\mathcal{\pi}_{\mu},\,\mathbf{x}\right)\,&=\,0,
            \end{split}
        \end{equation}
        \begin{equation}\label{eq: constraints on time derivatives}
            \begin{split}
                G\left(\phi^{\mu},\,\partial_{0}\phi^{\mu},\,\mathbf{x}\right)\,&=\,0.
            \end{split}
        \end{equation}
        The former type \eqref{eq: constraints on momenta} is constraints on canonical momenta, while the later type \eqref{eq: constraints on time derivatives} is constraints on time derivatives. Both two types lead to the fact that some time derivatives cannot be expressed in terms of canonical momenta. This is a pretty simple and obvious problem about linear equations in linear algebra. 
    \end{enumerate}
    \item Promotion of Classical Fields to Operators
    \newline In canonical quantization, the classical fields and their conjugate momenta are promoted to operators acting on a Hilbert space:
    \begin{equation}
        \begin{split}
            \phi^{\mu}\left(t,\mathbf{x}\right)\,&\rightarrow\,\hat{\phi}^{\mu}\left(t,\mathbf{x}\right),  \\
            \mathcal{\pi}_{\mu}\left(t,\mathbf{x}\right)\,&\rightarrow\,\hat{\mathcal{\pi}}_{\mu}\left(t,\mathbf{x}\right).
        \end{split}
    \end{equation}
    These operators are taken to be Hermitian for real scalar fields:
    \begin{equation}
        \begin{split}
            \hat{\phi}^{\dagger}\left(t,\mathbf{x}\right)\,=\,\hat{\phi}\left(t,\mathbf{x}\right),\;\hat{\mathcal{\pi}}^{\dagger}\left(t,\mathbf{x}\right)\,=\,\hat{\pi}\left(t,\mathbf{x}\right).
        \end{split}
    \end{equation}
    They act on a Hilbert space $\mathcal{H}$ whose basis states represent quantum many-particle states. 
    \item Equal-Time Commutation Relations
    \newline The key quantum postulate in canonical quantization is the imposition of equal-time commutation relations between the field operator and the canonical momentum operator, which is 
    \begin{equation}
        \begin{split}
            \comm{\hat{\phi}^{\mu}\left(t,\mathbf{x}\right)}{\hat{\mathcal{\pi}}_{\mu}\left(t,\mathbf{y}\right)}\,&=\,i\hbar\delta^{3}\left(\mathbf{x}-\mathbf{y}\right).
        \end{split}
    \end{equation}
    And all other equal-time 
\end{enumerate}





\end{document}
